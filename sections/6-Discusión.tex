\section{Discusión}

\subsection{Comparativa modelo estadístico}


\subsection{Entorno de entrenamiento y ejecución}

El proyecto ha sido desarrollado utilizando el \textit{IDE} \textit{PyCharm Community} con la versión de \textit{Python 3.9.13} y las dependencias \textit{Figura\ \ref{code:dependencies}}.

\begin{code}[]{title=Extracto de las dependencias del archivo 'pyproject.toml', label=code:dependencies}{bash}
    [tool.poetry.dependencies]    
    spectral = "^0.23.1"
    python = ">=3.9, <=3.10"
    pandas = "2.0.3"
    pillow = "^10.0.0"
    joblib = "^1.2.0"
    humanize = "^4.6.0"
    scikit-learn = "1.3.2"
    lightgbm = "^4.0.0"
    xgboost = "2.0.0"
    seaborn = "0.13.0"
    openpyxl = "^3.1.2"
    sdv = "^1.2.1"
    pyinstaller = "6.0.0"
    imbalanced-learn = "^0.11.0"
    watchdog = "^3.0.0"
    xlsxwriter = "^3.1.9"
\end{code}

Para la gestión de las dependecias del proyecto, hemos utilizado el gestor de dependencias \textit{Poetry}, el cual nos ha permitido tener un entorno de ejecución y entrenamiento aislado del resto de dependencias del sistema, evitando problemas de compatibilidad entre versiones de paquetes y facilitándonos la compilación de una aplicación de escritorio.

Esta aplicación de escritorio consiste en un ejecutable que permite al usuario, de forma gráfica, seleccionar un directorio donde irán entrando imágenes \gls{bil} para ser procesadas. Una vez se haya procesado la imagen y predicho la contaminación de los granos, habría que, en una fase posterior, conectar la aplicación con un sistema real y comunicar los resultados a un sistema de control, para expulsar los granos contaminados que se detecten. Para ello, habría que tener en cuenta varios factores como por ejemplo valorar el tiempo de predicción para saber la velocidad a la que podría ir la cinta con grano, el espacio necesario en disco, la captura de imágenes en tiempo real, el coste de las predicciones, la posibilidad de realizar nuevos entrenamientos y cambiar el modelo, etc.


