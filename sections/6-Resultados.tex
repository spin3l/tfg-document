\section{Conclusiones}

Hemos cumplido con los objetivos marcados del proyecto, aprendiendo las bases del \gls{ml} y aplicándolas a un problema real, siendo capaces de detectar granos contaminados en las imágenes \gls{bil} de forma automática.

A medida que hemos ido probando diferentes metodologías a lo largo del proyecto, hemos visto la importancia de la elección tanto de las herramientas de preprocesado como de los modelos.
Una posible mejora que no hemos utilizado en este proyecto, en parte porque el objetivo que tenía con el era aprender las bases del \gls{ml} utilizando \textit{sklearn}, es la utilización de redes neuronales, las cuales suelen funcionar mejor.

Otra posible mejora sería la utilización de un \textit{pipeline} de \textit{sklearn} para la realización tanto de las transformaciones de los datos, como del entrenamiento de los modelos, agilizando el proceso de entrenamiento y validación de los modelos. 

También, en el paso de detectar \textit{outliers}, podríamos en lugar de eliminar los granos con valores atípicos, interpolar esos valores para no perder demasiados datos ni trabajar con datos ``crudos''.