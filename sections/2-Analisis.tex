\section{Análisis del problema}

 Para ello, tenemos una base de datos de imágenes hiperespectrales con formato \acrshort{bil} (\textbf{figura\ \ref{fig:hyperspectral_image}}), de la cual hemos extraído los píxeles que forman los granos, por otro lado, tenemos otra base de datos con el estado de contaminación de estos mismos granos. 


\subsection{NIR-HSI}

\gls{nir-hsi} es una tecnología rápida, no destructiva y precisa que nos permite hacer inspecciones de calidad, la cual ha demostrado su potencial en los últimos años\ \cite{Applicat5:online}. Es una técnica de imagen química basada en la espectroscopia de reflectancia (la luz reflejada por los materiales), la cual es capaz de caracterizar compuestos orgánicos y algunos minerales\ \cite{NIRHyper23:online}.

Como hemos comentado en la introducción, nuestro objetivo es conseguir un modelo que prediga lo mejor posible qué granos de una \gls{imagen hiperespectral} contienen granos contaminados con \acrshort{don}.


\subsection{Separación del grano contaminado}\ \label{sec:separacion}

El valor de contaminación \gls{don} de un grano lo obtenemos de hacer un proceso químico que no entra dentro del propósito de este proyecto. Lo único que nos interesa es que el valor de la contaminación es un valor real. Además, sabemos que desde el laboratorio se considera que un grano está contaminado a partir de una concentración de \(1250 \mu g/kg\). De esta forma, aunque es un valor real, podemos utilizarlo como tal o considerar solamente si está contaminado o no. Es decir, considerarlo como un problema de \gls{clasificación} o de \gls{regresión}.

Aunque tenemos el valor continuo de contaminación, podemos reemplazarlo directamente por una columna booleana que indique si está contaminado. De esta forma pasaríamos de un problema de regresión, el cual nos permite predecir valores continuos, a uno de clasificación para predecir valores discretos, si está contaminado o no.

