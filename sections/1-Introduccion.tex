\section{Introducción}

Nuestro proyecto está enfocado en detectar un problema de seguridad alimentaria determinando la contaminación por \acrfull{don}, una toxina producida por hongos del género Fusarium, tales como Fusarium graminearum y Fusarium culmorum. 

Es interesante como sociedad que los alimentos que consumimos directa o indirectamente pasen un control de calidad, pues dicha toxina tiene efectos negativos sobre la salud humana y animal\ \cite{https://doi.org/10.2903/j.efsa.2017.4718}.


\subsection{Motivación}

Se nos presentó la posibilidad de aportar de forma remunerada en este proyecto y además utilizarlo como trabajo de final de grado. Otro de los principales motivantes ha sido la utilización de Python, pues este proyecto me ha facilitado la exploración de nuevas librerías y profundización de algunas librerías que hemos utilizado en clase en proyectos de \gls{ml}.


\subsection{Objetivos}

El principal objetivo es realizar un proceso automatizado de análisis de imágenes hiperespectrales, el cual nos muestre qué granos de trigo están contaminados por la toxina. Emulando una parte del proceso productivo en una cinta transportadora, es decir, tratando de optimizar el tiempo de ejecución de los análisis y preparando el programa para ser ejecutado durante un tiempo indefinido. 

Los pasos que hemos discernido del objetivo principal para realizar proyecto son los siguientes:

\begin{enumerate}
    \item Entender los datos de los resultados de laboratorio.
    \item Entender el existente proceso de análisis de imágenes hiperespectrales.
    \item Tratar de mejorar el actual proceso de análisis de las imágenes.
    \item Entender el actual proceso de entranamiento y análisis de modelos de \gls{ml}.
    \item Tratar de mejorar el preprocesado de los datos, tanto como el entrenamiento de nuevos modelos para tratar de obtener mejores resultados.
    \item Entrenar buenos modelos de \gls{ml} que, según las imágenes hiperespectrales, sean capaces de predecir si un grano está contaminado. 
    \item Una vez entrenados los modelos básicos, tratar de ajustarlos a los datos con \textit{Hyperparameter tuning}.
    \item Añadir nuevos modos de ejecución del programa para emular la carga continua de imágenes que habría en un caso de uso real, además de prepararlo para que se pueda ejecutar por un tiempo indefinido como en un caso real, guardando los resultados de las predicciones.
\end{enumerate}

\subsection{Estructura del documento}

El documento tiene un orden lógico. Primeramente, introduciremos el problema real indicando cómo se toman los datos en laboratorio, cómo analizaremos estos datos y cómo la ley ampara este ámbito. 

A continuación, explicaremos los conceptos básicos de \gls{ml} que hemos utilizado para la realización del proyecto, pondremos algunos ejemplos y comentaremos las principales fases de un proyecto de \gls{ml}. 

Antes de diseñar el proyecto, habiendo analizado los datos de partida, probaremos diferentes formas de preprocesarlos para que los modelos entrenados con estos tengan mejores resultados. 

Por último, analizaremos y compararemos los resultados, tratando de ver la viabilidad de aplicarlos en una aplicación de tiempo real.

Además, podemos encontrar algunas \hyperref[sec:additional-definitions]{definiciones adicionales}, \hyperref[sec:acronyms]{acrónimos} y el \hyperref[sec:anex]{anexo}.