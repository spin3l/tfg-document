\section{Introducción}

Las micotoxinas son toxinas naturales derivadas del metabolismo secundario de mohos micotoxigénicos, que se pueden encontrar en alimentos y materias primas. Una de ellas,
el \acrfull{don}, se encuentra en el campo predominantemente en granos de cereales como el trigo. Por este motivo, el deoxinivalenol se encuentra frecuentemente en
alimentos a base de cereales lo que, unido a la elevada ingesta de trigo típica de la dieta española, hace que la exposición de la población sea significativa. En los años
en que la meteorología es especialmente adversa los cultivos de todo el mundo muestran altos niveles de contaminación. Además, en los últimos años existe una creciente
evidencia de un aumento en la incidencia de micotoxinas como resultado del cambio climático.

Actualmente, las empresas procesadoras de cereales basan su autocontrol en el análisis de muestras de un determinado número de lotes mediante técnicas inmunocromatográficas
rápidas, de forma que se rechazan los lotes no conformes. Los lotes que, una vez analizados, superan el límite legal \(1250 \mu g/kg\), en general, se desvían íntegramente
a la alimentación animal. En el sector de la alimentación animal, el problema se hace más evidente, ya que el consumo de piensos contaminados se evidencia en una menor
ganancia de peso del ganado y el impacto en otros parámetros productivos y de salud animal.


\subsection{Motivación}


A nivel de proyecto, el desafío que se plantea en este proyecto se centra en las operaciones de selección y clasificación de granos mediante \acrfull{nir-hsi}, previas a la transformación, que
de ser efectivas podrían aplicarse a lotes que superen los \(1250 \mu g/kg\), con la intención de devolver dichos lotes a la cadena alimentaria, una vez se ha separado
la fracción más contaminada. De esta manera, la fracción que podría destinarse al consumo humano sería posiblemente mayoritaria, y el sistema de producción de alimentos,
en su conjunto, más sostenible.

Personalmente, se nos presentó la posibilidad de aportar de forma remunerada en este proyecto y además utilizarlo como trabajo de final de grado. Otro de los principales motivantes 
ha sido la utilización de Python, pues este proyecto me ha facilitado la exploración de nuevas librerías y profundización de algunas que hemos utilizado en 
clase en proyectos de \acrfull{ml}.


\subsection{Objetivos}

A nivel de proyecto, el objetivo es claro. La obtención de datos suficientes para construir un modelo de predicción robusto, la validación de la técnica en
condiciones de laboratorio con un número suficiente de muestras reales y el desarrollo de software para la detección en tiempo real de granos contaminados. 

Para ello, hemos separado esta tarea (desarollo de software para la detección en tiempo real de granos contaminados) en diferentes pasos. 

\begin{enumerate}
    \item Entender los datos de los resultados de laboratorio.
    \item Entender el existente proceso de análisis de imágenes hiperespectrales.
    \item Tratar de mejorar el actual proceso de análisis de las imágenes.
    \item Entender el actual proceso de entranamiento y análisis de modelos de \gls{ml}.
    \item Tratar de mejorar el preprocesado de los datos, tanto como el entrenamiento de nuevos modelos para tratar de obtener mejores resultados.
    \item Entrenar buenos modelos de \gls{ml} que, según las imágenes hiperespectrales, sean capaces de predecir si un grano está contaminado. 
    \item Una vez entrenados los modelos básicos, tratar de ajustarlos a los datos con \textit{Hyperparameter tuning}.
    \item Añadir nuevos modos de ejecución del programa para emular la carga continua de imágenes que habría en un caso de uso real, además de prepararlo para que se 
    pueda ejecutar por un tiempo indefinido como en un caso real, guardando los resultados de las predicciones.
\end{enumerate}

\subsection{Estructura del documento}